\documentclass{bsuir}

\departmentlong{инженерной психологии и эргономики}
\workcode{9}
\worktitle{Рекурсия}
\titlepageyear{2024}

\begin{document}

\maketitle{
    Проверил:\\
    Кабариха В.А.\\
    ~
}{
    Выполнил:\\
    Бородин А.Н.\\
    гр. 310901
}

\textbf{Цель}: Сформировать знания и умения по работе с подпрограммами,
приобрести навыки написания программ с использованием рекурсивных функций.

\section*{Задание 2}

Для данного N вычислить значение выражения, используя рекурсию:
$P=\sqrt{2+\sqrt{4+\sqrt{6+\ldots+\sqrt{2N}}}}$

\makelisting{1.cc}

\makelisting{1-1.txt}[Вывод программы (легкопроверяемый)]

\makelisting{1-2.txt}[Вывод программы]

\textbf{Вывод}: Освоено использование подпрограмм и рекурсия в языке C++.

\end{document}
