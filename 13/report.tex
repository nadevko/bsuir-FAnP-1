\documentclass[variant=labwork]{bsuir}

\departmentlong{инженерной психологии и эргономики}
\workcode{13}
\worktitle{Обратная польская запись}
\titleleft{
    Проверил:\\
    Кабариха В.А.\\
    ~
}
\titleright{
    Выполнил:\\
    Бородин А.Н.\\
    гр. 310901
}
\titlepageyear{2024}

\begin{document}

\maketitle

\textbf{Цель}: Сформировать знания и умения по работе с подпрограммами,
приобрести навыки написания программ с использованием обратной польской записи
(ОПЗ).

\section*{Задание 1}

Описать функции, которая вычисляет значение заданного выражения. Входные данные.
В первой строке содержит обратную польскую запись арифметического выражения. Все
операнды целые положительные числа. Выходные данные. Вывести результат
вычисления ОПЗ.

Технические требования. Используются знаки операций: +, -, *, /.

\makelisting{1.cc}

\makelisting{1.txt}[Вывод программы]

\section*{Задание 2}

На вход программы поступает выражение, состоящее из односимвольных
идентификаторов и знаков арифметических действий. Требуется преобразовать это
выражение в обратную польскую запись или же сообщить об ошибке.

\makelisting{2.cc}

\makelisting{2.txt}[Вывод программы]

\makelisting{lib/shunting.cc}[shunting.cc]

\textbf{Вывод}: Изучена обратная польская запись, улучшены навыки использование
подпрограмм.

\end{document}
