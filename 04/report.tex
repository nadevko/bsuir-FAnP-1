\documentclass{bsuir}

\departmentlong{инженерной психологии и эргономики}
\workcode{4}
\worktitle{Массивы}
\titlepageyear{2023}

\begin{document}

\maketitle{
    Проверила:\\
    Василькова А.Н.\\
    ~
}{
    Выполнил:\\
    Бородин А.Н.\\
    гр. 310901
}

\textbf{Цель}: Сформировать навыки и умения обработки структурированных типов
данных, организованных в виде некоторой линейной последовательности, а также
организованных в виде матрицы.

\section*{Задание 1.2 (Одномерные массивы)}

В одномерном массиве, состоящем из n вещественных элементов, вычислить
произведение элементов с четными номерами.

\makelisting{1.cc}

\makelisting{1.txt}[Вывод программы]

\makesvg{1.drawio.svg}[Схема алгоритма]

\section*{Задание 2.2 (Двумерные массивы)}

Дан двумерный массив целых чисел. Вычислить сумму элементов, расположенных на
главной и побочной диагоналях.

\makelisting{2.cc}

\makelisting{2.txt}[Вывод программы]

\makesvg{2.drawio.svg}[Схема алгоритма]

\textbf{Вывод}: Освоены n-мерные массивы в языке C++.

\end{document}
