\documentclass{bsuir}

\departmentlong{инженерной психологии и эргономики}
\workcode{3}
\worktitle{Циклические алгоритмы}
\titlepageyear{2023}

\begin{document}

\maketitle{
    Проверила:\\
    Василькова А.Н.\\
    ~
}{
    Выполнил:\\
    Бородин А.Н.\\
    гр. 310901
}

\textbf{Цель}: Сформировать умения разрабатывать программы с использованием
операторов выбора, цикла и передачи управления.

\section*{Задание 1.2 (For)}

Найти все трехзначные числа, которые кратны 2 и 4.

\makelisting{1.cc}

\makelisting{1.txt}[Вывод программы]

\makesvg{1.drawio.svg}[Схема алгоритма]

\section*{Задание 2.2 (While)}

Вычислить сумму ряда с заданной степенью точности $\alpha$:\\$\displaystyle
\sum_{n=1}^{\infty}(-1)^n*\frac{1}{3n^2}$, $\alpha=0,0001$

\makelisting{2.cc}

\makelisting{2.txt}[Вывод программы]

\makesvg{2.drawio.svg}[Схема алгоритма]

\textbf{Вывод}: Освоены циклы с предусловием (while), с постусловием (do\ldots
while) и параметром (for) языка C++.

\end{document}
