\documentclass[variant=labwork]{bsuir}

\departmentlong{инженерной психологии и эргономики}
\workcode{15}
\worktitle{Хеширование}
\titleleft{
    Проверил:\\
    Кабариха В.А.\\
    ~
}
\titleright{
    Выполнил:\\
    Бородин А.Н.\\
    гр. 310901
}
\titlepageyear{2024}

\begin{document}

\maketitle
\mainmatter

\textbf{Цель}: Сформировать знания и умения по работе с подпрограммами,
приобрести навыки написания программ с использованием хеш-функций.

\section*{Общее задание}

Составить хеш-функцию в соответствии с заданным вариантом и проанализировать ее.
При необходимости доработать хеш-функцию. Используя полученную хеш-функцию
разработать на языке программирования C++ программу, которая должна выполнять
следующие функции:

\begin{itemize}
    \item создавать хеш-таблицу;
    \item добавлять элементы в хеш-таблицу;
    \item просматривать хеш-таблицу;
    \item искать элементы в хеш-таблице;
    \item удалять элементы из хеш-таблицы.
\end{itemize}

\section*{Описание хеш-функции}

Хеш-функция основана на возведении суммы кодов символов ключа в квадрат и
извлечение из полученного квадрата нескольких средних цифр. При этом коды
символов умножаем на частное кода и произведения тройки на порядковый номер
символа в ключе (1ч6). Звучит убого, вот так выглядит формула суммы:

$\displaystyle(\sum_{i=1}^{n}\frac{v^2}{3i})^2$, где $v$ - код символа с
индексом \textquote{$i$};

Возведенная в квадрат сумма колеблется от 7997584 до 22781529, а это семизначное
или восьмизначное число. Для адресации сегментов хеш-таблицы необходимо
четырехзначное число, не превышающее 2000. Откинем у квадрата суммы 2 первых и
два последних разряда, так у нас получится трехзначное или четырехзначное число.
Для того, чтобы адрес не превысил максимально допустимый адрес 1999, будем брать
остаток от деления на 2000 до тех пор, пока он не попадет в нужный диапазон.

\section*{Экспериментальный анализ хеш-функции}

Экспериментальное исследование проводится следующим образом:

\begin{itemize}
    \item формируются случайным образом ключи заданного формата в количестве,
          превышающем количество сегментов хеш-таблицы в 2\ldots 3 раза;
    \item для каждого сформированного ключа вычисляется хеш-функция, и
          подсчитывается, сколько раз вычислялся адрес того или иного сегмента
          хеш-таблицы.
\end{itemize}

\makelisting{1.cc}

\makelisting{1.txt}[Фрагмент вывода программы]

\textbf{Вывод}: Освоены принципы хеширования и хеш-таблицы.

\end{document}
