\documentclass[variant=labwork]{bsuir}

\departmentlong{инженерной психологии и эргономики}
\workcode{2}
\worktitle{Разветвляющиеся алгоритмы}
\titleleft{
    Проверила:\\
    Василькова А.Н.\\
    ~
}
\titleright{
    Выполнил:\\
    Бородин А.Н.\\
    гр. 310901
}
\titlepageyear{2023}

\begin{document}

\maketitle
\mainmatter

\textbf{Цель}: Изучить основы языка C++, необходимые для написания разветвлённых
алгоритмов.

\section*{Задание 2}

Из трех данных чисел выбрать наименьшее.

\makelisting{1.cc}

\makelisting{1.txt}[Вывод программы]

\makesvg[Схема алгоритма]{1.drawio.svg}

\textbf{Вывод}: Освоен условный оператор if (if-else) языка C++.

\end{document}
