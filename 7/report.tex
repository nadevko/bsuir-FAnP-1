\documentclass{bsuir}

\departmentlong{инженерной психологии и эргономики}
\workcode{7}
\worktitle{Структуры и файлы}
\titlepageyear{2023}

\begin{document}

\maketitle{
    Проверила:\\
    Василькова А.Н.\\
    ~
}{
    Выполнил:\\
    Бородин А.Н.\\
    гр. 310901l
}

\textbf{Цель}: Cформировать навыки и умения обработки структурированных типов
данных, организованных в виде структур и файлов.

\section*{Задание 2}

Описать структуру с именем $\mathrm{STUDENT}$, содержащую следующие поля:
\begin{itemize}
    \item фамилия и инициалы
    \item номер группы
    \item успеваемость (массив из пяти элементов)
\end{itemize}
Написать программу, выполняющую следующие действия:
\begin{itemize}
    \item ввод с клавиатуры данных
    \item Упорядочивание по алфавиту
    \item Запись в файл
\end{itemize}

\makelisting{1.cc}

\makelisting{1.txt}[Вывод программы]

\makelisting{1.ini}[students.ini]

\textbf{Вывод}: Освоено создание структур (struct) и процесс чтения-записи
файлов в языке C++.

\end{document}
