\documentclass{bsuir}

\departmentlong{инженерной психологии и эргономики}
\workcode{11}
\worktitle{Списки. Стек}
\titlepageyear{2024}

\begin{document}

\maketitle{
    Проверил:\\
    Кабариха В.А.\\
    ~
}{
    Выполнил:\\
    Бородин А.Н.\\
    гр. 310901
}

\textbf{Цель}: Cформировать умения и навыки написания программ с использованием
стека, списков.

\section*{Задание 2}

Опишите и постройте с помощью двумерного массива $\mathrm{Sps}$ линейный
однонаправленный список из семи целых чисел и сделайте этот список пустым. После
этого добавьте в список шесть элементов $1$, $3$, $5$, $7$, $9$, $11$, затем
найдите указатель на элемент 9 и удалите этот элемент. В конце работы со списком
вставьте после элемента со значением $11$ элемент со значением $13$,
предварительно отыскав указатель на элемент со значением $11$, а элемент со
значением $15$ вставьте после элемента со значением $3$. Результаты как
промежуточных, так и конечных результатов отобразить на экране.

\makelisting{1.cc}

\makelisting{1.txt}[Вывод программы]

\textbf{Вывод}: изучены списки и стек.

\end{document}
