\documentclass{bsuir}

\departmentlong{инженерной психологии и эргономики}
\workcode{14}
\worktitle{Бинарное дерево}
\titlepageyear{2024}

\begin{document}

\maketitle{
    Проверил:\\
    Кабариха В.А.\\
    ~
}{
    Выполнил:\\
    Бородин А.Н.\\
    гр. 310901
}

\textbf{Цель}: Сформировать знания и умения по работе с подпрограммами,
приобрести навыки написания программ с использованием бинарных деревьев.

\section*{Задание 2}

Опишите класс --- дерево, необходимое для решения задачи, указанной в вашем
варианте задания, и реализуйте его методы.

Продемонстрируйте работу основных методов работы с деревом: построение, вывод,
обход.

Составьте программу решения задачи, указанной в вашем варианте задания.

На основе операции обхода сверху вниз реализовать операцию, определяющую,
подобны ли два бинарных дерева (два бинарных дерева подобны, если они оба пусты,
либо их левые и правые поддеревья подобны).

\makelisting{1.cc}

\makelisting{1.txt}[Вывод программы]

\textbf{Вывод}: Освоены бинарные деревья.

\end{document}
