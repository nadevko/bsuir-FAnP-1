\documentclass{bsuir}

\departmentlong{инженерной психологии и эргономики}
\departmentshort{~}
\worktitle{
    Тестовая работа по предмету\\
    \textquote{Конструирование программного обеспечения}
}
\titleleft{
    Проверила:\\
    Василькова А.Н.\\
    ~\\
    ~
}
\titleright{
    Выполнили:\\
    Бородин А.Н.\\
    Яковцов Д.В.\\
    гр. 310901
}
\titlepageyear{2024}

\newcommand{\csharp}{C{\liberationrm\#}}

\departmentlong{Кафедра инженерной психологии и эргономики}

\begin{document}

\maketitle

\chapter{Пункт 8}

\section{Рекурсивные функции}

\makeimage{5.png}[Рекурсия на примере чисел Фибоначчи]

Рекурсивная функция -- это такая функция, которая вызывает себя из себя же
самой.

Базовый случай -- каждая рекурсивная функция должна иметь базовый случай,
который предотвращает бесконечные вызовы, обеспечивая выход из рекурсии.

Рекурсивный вызов -- функция выполняет рекурсивный вызов самой себя с измененным
набором параметров, приближающихся к базовому случаю.

Стек вызовов -- каждый рекурсивный вызов добавляет новый слой в стек вызовов,
сохраняя состояние текущего вызова до тех пор, пока не будет достигнут базовый
случай.

Важно помнить, что рекурсивные функции могут быть менее эффективными по памяти и
времени выполнения по сравнению с итеративными аналогами из-за накладных
расходов на стек вызовов. Однако, они могут предложить более чистый и понятный
код для некоторых типов задач.

\section{Локальные функции}

\makeimage{4.png}[Локальная функция и ее эффект]

Локальные функции -- это методы, которые объявляются внутри других
методов и могут вызываться только из своего содержащего метода. Они
предоставляют удобный способ инкапсулировать код, который необходим только в
рамках одного метода, делая его более читаемым и поддерживаемым.

Область видимости -- локальные функции могут быть вызваны только внутри метода,
в котором они объявлены.
    
Места объявления -- можно объявлять локальные функции в методах,
конструкторах, методах доступа к свойствам и событиям, анонимных методах и
лямбда-выражениях.
    
Модификаторы -- Локальные функции могут использовать модификаторы async, unsafe,
static, но не могут иметь модификаторы доступа, так как они всегда являются
приватными.
    
Статические локальные функции -- Статические локальные функции не могут
обращаться к локальным переменным или состоянию экземпляра содержащего их
метода.
    
Исключения -- Локальные функции могут обрабатывать исключения, возникающие
внутри них, что может быть полезно для немедленной обработки ошибок.

Локальные функции улучшают структуру кода и его понимание, так как они
ограничивают использование некоторых действий только внутри контекста, где они
действительно нужны. Это также помогает предотвратить непреднамеренное
использование метода в других частях класса или структуры.

\section{Конструкция switch}

\makeimage{1.png}[switch case]

\makeimage{2.png}[switch case с goto]

\makeimage{3.png}[switch expression]

Конструкция switch/case оценивает некоторое выражение и сравнивает его значение
с набором значений. И при совпадении значений выполняет определенный код. конце
каждого блока саве должен ставиться один из операторов перехода: break, goto
case, return или throw. Как правило, используется оператор break. При его
применении другие блоки case выполняться не будуть. Однако если мы хотим, чтобы,
наоборот, после выполнения текущего блока case выполнялся другой блок саѕе, то
мы можем использовать вместо break. Конструкция switch позволяет возвращать
некоторое значение. Для возвращения значения в блоках саве может применятся
операrop return.

\section{Перечисления enum}

\makeimage{6.png}[enum]

\makeimage{7.png}[enum с указанием типа]

\makeimage{8.png}[enum с указанием значений]

Тип перечисления (или тип enum) это тип значения, определенный. набором
именованных констант применяемого целочисленного типа. Чтобы определить тип
перечисления, используйте ключевое слово enam 

По умолчанию связанные значения констант элементов перечисления имеют тип int.
Они начинаются с нуля и увеличиваются на единицу в соответствии с порядком
текста определения. Вы можете явно указать любой другой целочисленный тип в
качестве базового типа перечисления. Вы можете также явно указать
соответствующие значения констант, как показано в следующем примере:

\end{document}
