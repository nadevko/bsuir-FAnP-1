\documentclass{bsuir}

\departmentlong{инженерной психологии и эргономики}
\workcode{8}
\worktitle{Функции}
\titlepageyear{2023}

\begin{document}

\maketitle{
    Проверила:\\
    Василькова А.Н.\\
    ~
}{
    Выполнил:\\
    Бородин А.Н.\\
    гр. 310901l
}

\textbf{Цель}: Сформировать навыки и умения обработки структурированных типов
данных, организованных в виде функций.

\section*{Задание 1.2 (Одномерные массивы)}

Дано натуральное число $m$. Укажите все тройки натуральных чисел $x$, $y$ и $z$,
удовлетворяющие следующему условию: $m = x^3 + y^3 + z^3$.

\makelisting{1.cc}

\makelisting{1.txt}[Вывод программы]

\section*{Задание 2.2 (Двумерные массивы)}

Используя перегрузку методов, создайте программу
\begin{enumerate}
    \item для сложения целых чисел;
    \item для сложения комплексных чисел
\end{enumerate}

\makelisting{2.cc}

\makelisting{2-1.txt}[Вывод программы (целые)]

\makelisting{2-1.txt}[Вывод программы (комплексные)]

\textbf{Вывод}: Освоено объявление функций, их шаблоны, типизация и указатели в
языке C++.

\end{document}
